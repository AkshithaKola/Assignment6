\documentclass{beamer}
\usetheme{CambridgeUS}

\title{Assignment 6 : Papoulis Textbook }
\author{Akshitha Kola}
\date{\today}
\logo{\large \LaTeX{}}

\usepackage{amsmath}
\setbeamertemplate{caption}[numbered]{}
\providecommand{\pr}[1]{\ensuremath{\Pr\left(#1\right)}}
\providecommand{\cbrak}[1]{\ensuremath{\left\{#1\right\}}}

\begin{document}

\begin{frame}
    \titlepage 
\end{frame}

\logo{}

\begin{frame}{Outline}
    \tableofcontents
\end{frame}

\section{Question}
\begin{frame}{Question}
    \begin{block}{Chapter 3 example 3.17}
    A and B plays a series of games where the probability of winning p in a single play for A is unfairly kept at less than $\frac{1}{2}$. However, A gets to choose in advance the total number of plays. To win the whole game one must score more than half the plays. If the total number of plays is to be even, how many plays should A choose?
    \end{block}
\end{frame}

\section{Solution}
\begin{frame}{Solution}
\frametitle{Solution}
As the total number of plays to be played is even, let the total number of plays be 2n. Let $X_{k}$ be the number of plays A wins out of 2n plays.
Then $P(X_{k}) = \begin{pmatrix} 2n\\k\end{pmatrix} p^{k}q^{2n-k}$

Let $P_{2n}$ denote the event that A wins in 2n plays.
\begin{align}
P_{2n} &= P\left(\bigcup\limits_{k=n+1}^{2n} X_{k}\right) \\
&= \sum_{k=n+1}^{2n} P(X_{k})\\
&= \sum_{k=n+1}^{2n} \begin{pmatrix} 2n\\k\end{pmatrix} p^{k}q^{2n-k}
\end{align}
\end{frame}

\begin{frame}
As the required number of plays is 2n 
 $P_{2n-2}\leq P_{2n}\geq P_{2n+2}$
\begin{align}
P_{2n+2} &= \sum_{k=n+2}^{2n+2} \begin{pmatrix} 2n+2\\k\end{pmatrix} p^{k}q^{2n+2-k} = (p+q)^{2n+2} = (p+q)^{2n}(p+q)^{2}  \\
&= \left\lbrace\sum_{k=n+1}^{2n} \begin{pmatrix} 2n\\k\end{pmatrix} p^{k}q^{2n-k}\right\rbrace (p^{2}+ 2pq + q^{2}) \\
\end{align}
From the above equation we get a identity after equating the powers of p and q.

\begin{align}
P_{2n+2} = P_{2n} + \begin{pmatrix} 2n\\n\end{pmatrix} p^{n+2}q^{n} - \begin{pmatrix} 2n\\n+1\end{pmatrix} p^{n+1}q^{n+1}
\end{align}
\end{frame}

\begin{frame}
If 2n is optimum, then from the inequality we get
\begin{align}
\begin{pmatrix} 2n\\n+1\end{pmatrix} p^{n+1}q^{n+1} \leq \begin{pmatrix} 2n\\n\end{pmatrix} p^{n+2}q^{n} \\
\implies nq \leq (n + 1)p \implies n(q-p) \leq p \implies  n \leq \frac{p}{1-2p}
\end{align}
and
\begin{align}
\begin{pmatrix} 2n-2\\n-1\end{pmatrix} p^{n+1}q^{n-1} \leq \begin{pmatrix} 2n-2\\n\end{pmatrix} p^{n}q^{n} \\
\implies np \leq (n - l)q \implies n(q-p) \geq q \implies  n \geq \frac{q}{1-2p}
\end{align}
\end{frame}

\begin{frame}
From (9) and (11) we get 
\begin{align}
\frac{1}{1-2p} - 1 \leq 2n \leq \frac{1}{1-2p} + 1
\end{align}
$\therefore$ This means 2n is the even number nearest to $\frac{1}{1-2p}$
\end{frame}
\end{document}
